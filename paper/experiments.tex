\section{Experiments}
\label{sec:design}

    \subsection{Permissions Misuse}
    
        Many android applications misuse permissions that they
        allow their application to possess. Often, certain
        permissions are granted to an app that doesn't use them
        in the first place. This practice is dangerous: it can 
        leave vulnerabilities within the application that should
        not be possible based on the actual functionality of the
        application. For example, it would be easy for a 
        developer to add functionality to an app that requires a
        certain permission, and then leave the permission in after
        reverting their previous changes. 
        
        Another common theme seen within applications
        is the combination of two permissions that can create a 
        dangerous comibination. For example, a malicious 
        application with the permission combination of
        CAMERA and RECORD\_AUDIO would allow the application
        to have access to a devices camera and microphone, tools
        that would be able to perform serious invasions of privacy. 
        Another example could be seen with INTERNET and 
        ACCESS\_FINE\_LOCATION, which would grant an application the
        ability to track a device's physical location (and 
        subsequently serve as a tool for stalking).

        The ability of permissions to allow an application within
        reach of sensitive data means that users should be prompted
        whether they wish to allow certain permissions to be enabled.
        However, many applications do not give users the chance to 
        make this decision; this choice is a significant breach of trust 
        between an application and its users.

        All of these developer mistakes constitute permission vulnerabilities
        within android applications. For our first experiment, we
        will be testing these common misuses of permissions. First, 
        the experiment will test whether apps utilize all of the
        permissions that they request. Second, the experiment will 
        search for the use of dangerous permission combinations
        implemented in the app. Finally, the experiment will test whether
        users are prompted to explicity give their consent to every
        permission used.
        
        Setup:

    \subsection{Trust Managers and Error Handlers}

        Trust Managers are put in place to verify the authenticity of a 
        remote server. To do this, many android built-in trust managers 
        are implemented to securely verify a server's certificate.
        However, the built-in X509TrustManager class allows the complete 
        override of the server verification process, potentially endangering
        an application if implemented incorrectly.

        Many times, developers will avoid the built-in trust manager in an
        effort to take shortcuts around the correct implementation (whether
        this be for convenience or lack of experience). This practice is often
        carried out by implementing the checkServerTrusted() and getAcceptedIssuers() 
        functions in a way that configures the hostname verifier to trust all X.509
        certificates. By doing this, developers expose their application to danger; 
        third parties may attempt a Man-in-the-Middle attack on network traffic
        from the application, compromising a user's network data if successful.

        TALK ABOUT ERROR HANDLING HERE

        This experiment will be testing whether or not an app overrides a built-in
        trust manager or error handler to forgo either method's intended purpose of
        correctly verifying certificates.

        Setup:

    \subsection{AllowAllHostnameVerifier}
    
        The HostnameVerifier interface within Android Studio is responsible for the 
        verification of the hostname within the server being connected to, making sure the hostname within
        the server's certificate matches the one seen in the server the client is attempting
        to connect to. 

        A vulnerability arises when the developer attempts to shortcut the hostname verification
        process (similar to Experiment 2), resulting in an ineffective verification process.
        Specifically, many developers use 

    \subsection{Mixed use SSL}

    \subsection{addJavascriptInterface Method}
