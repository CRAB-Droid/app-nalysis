\section{Findings and Future Work}
\label{sec:relwork}

%Related Work (``somewhat related'' work goes here; directly related work
%goes into the Introduction)~\cite{dsd13}.
\subsection{Analysis of Results}

\subsection{Future Work}
If we were to continue this research in the future, we would want to explore some of the following areas. 

First and foremost, we would like to have more though experiments. Our experiments currently utilize the Androguard library in order to parse the apk files 
and find simple string matches. One example is from experiment two, where we search for overridden built-in methods. This experiment is fairly simplistic 
and could be expanded upon into something where the method internals are also automatically checked to view if they are forgoing original intended use, making the user vulnerable.

Additionally, we would like to expand our set of Android applications to test on. If we were able to test on a larger set of apps, this would allow us to draw more accurate 
conclusions and notice more common trends in regards to vulnerabilities. This would allow us to develop a more comprehensive tool and experiments. 

Finally, instead of just utilizing static analysis, breaking into the domain of dynamic analysis could prove to be beneficial in seeing how these vulnerabilities can be exploited in real time. 
This would give us a chance to see how our predictions of vulnerabilities holds in a test environment of the application running. The results could prove valuable to enforce our findings and 
potentially find more vulnerabilities that were not found in static analysis. 
