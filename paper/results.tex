\section{Results}
\label{sec:discussion}

\begin{enumerate}
    \item Experiment 1: Permission Misuse
    
    90 of the 97 apps contained permissions that were 
    unused (and thus unnecessarily added). The average number 
    of unused permissions used within all apps was 6.01, 
    and when only considering apps with at least one 
    unused permission, this number rises to 6.48.

    62 of the 97 apps were identified to be using a dangerous
    combination of permissions. The average number of dangerous permission
    combinations within all apps totaled 1.21, with the number
    rising to 1.89 when only considering apps with at least one
    instance of a dangerous permission combination.

    All of the apps requested every permission they included, as is expected by Android applications. 

    False positives can arise for this experiment in situations where XXXX....
    Are there false positives for this??

    \item Experiment 2: Trust Managers and Error Handlers
    
    66 of the 97 apps contained overriden trust managers.
    The average number of overridden trust managers within the 
    apps totaled 3.06, with the number rising to 4.50 when only 
    considering apps with at least one instance of an overridden trust 
    manager.

    51 of the 97 apps contained overridden error handlers.
    The average number of overridden error handlers within the apps
    totaled 1.34, with the number rising to 2.55 when only considering apps
    with at least one instance of an overriden error handler.

    False positives can arise for this experiment in situations where a safely implemented override occurs (i.e. a developer creates a new, safe trust manager or error handler). It is also possible when a developer fully trusts a remote server, and feels they do not have to undergo the verification of a certificate.

    \item Experiment 3: AllowAllHostnameVerifier
    
    20 of the 97 apps contained the AllowAllHostnameVerifier class. 
    The average number of AllowAllHostnameVerifier uses within all
    apps totaled to 0.25, with the number rising to 1.26 when only
    considering apps with at least one instance of the class.

    False positives can arise in this experiment in situations where the developer trusts the host of a server being connected to, bypassing the need to check their authenticity.

    False positives also may include situations where this verifier is located in dead code or commented out.

    \item Experiment 4: Mixed use SSL
    
    90 of the 97 apps contained mixed use SSL. None of the apps 
    contained only HTTPS or HTTP usage, and 5 of the apps used no 
    URLs at all. The average number of HTTP URLs within the apps 
    totaled to 49.51, with this number rising to 52.77 when only 
    considering apps with at least one HTTP URL being used.

    False positives can arise in this experiment in situations where HTTP content being loaded by an HTTPS page is known and trusted by a developer.IFF on this 
          
    Another chance for a false positive arises when the script finds "http://" in dead code. 

    \item Experiment 5: addJavascriptInterface Method
    
    80 of the 97 apps contained improper handling of the addJavascriptInterface
    method. The average number of javascript vulnerabilites totaled to 
    3.67 per app, with the number rising to 4.75 when only considering
    apps with at least one vulnerability of this kind.

    False positives can occur in this experiment in situations where the addJavascriptInterface method is implemented (without any @JavascriptInterface annotations in the class), but the developer trusts the source of the content being used within the webview.

\end{enumerate}

%Discussion (discuss some of the important simplifying assumptions, and
%suggest possibilities for future work)

%Just put pure results stuff here, will do analysis in next section



