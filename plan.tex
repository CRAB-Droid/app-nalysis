\documentclass[10pt,twocolumn,pdftex]{article}
\usepackage[margin=1in]{geometry}
\usepackage{comment}
\usepackage{graphicx}
\usepackage{url}
\usepackage[pdftex,colorlinks=true,citecolor=black,filecolor=black,%
            linkcolor=black,urlcolor=black]{hyperref}
\usepackage{times}
%\usepackage{listings}
%\usepackage{fancyvrb}
%\usepackage{amsmath}
%\usepackage{amsthm}
%\usepackage{amssymb}

%\lstset{ % for our code environment
%    language={},
%    basicstyle=\ttfamily}
%\let\code\lstinline

\title{CSCI 420-03: Analysis Plan}
\author{Student Name, Student Name \\
\url{{student.name, student.name}@email.wm.edu}}
\date{}

% Simple macro to help with research questions
\newcounter{rqcounter}
\setcounter{rqcounter}{0}
% Usage: \newrq{label}{question}
\newcommand{\newrq}[2]{\noindent\refstepcounter{rqcounter}\textbf{RQ\arabic{rqcounter}:} {\em #2}\label{#1}}
% Usage: \rqref{label}
\newcommand{\rqref}[1]{\textbf{RQ\ref{#1}}}


\begin{document}
\maketitle

\section{Research Questions}

    \newrq{rq:req_perm}{Do apps use all of the permissions they request?}

    \newrq{rq:combo_perm}{Are any of the following dangerous permission combinations included? 

        \texttt{RECORD\_AUDIO \& INTERNET} (eavesdropping), 

        \texttt{ACCESS\_FINE\_LOCATION \& RECEIVE\_BOOT\_COMPLETED} (tracking), 

        \texttt{CAMERA \& INTERNET} (stalking),

        \texttt{SEND\_SMS \& WRITE\_SMS} (use phone as spam bot)

    }

    \newrq{rq:cert_ssl}{Do apps verify certifications incorrectly (Incorrect overrides of trust managers or error handlers methods)?} 

    \newrq{rq:host_ssl}{Do apps use improper Hostname verification?}

    \newrq{rq:deprecated_ssl}{Are apps using deprecated or vulnerable SSL protocols?}

    \newrq{rq:strip_ssl}{Do apps have mixed SSL use/are vulnerable to SSL Stripping attacks?}

    \newrq{rq:intents_interf}{Are sensitive data or mutable objects used in any implicit intents?}

    \newrq{rq:webview_interf}{Is trusted content loaded within any webviews? If displaying user-provided content, is data loaded into webviews sanitized?}

    \newrq{rq:js_interf}{Are applications that use webviews and \texttt{addJavascriptInterface()} correctly using the \texttt{@JavascriptInterface} annotation?}


\section{Hypotheses}
A high level description of how you plan to answer the research
questions, along with a list of hypotheses.



\section{Evaluation Plan}

A description of how you plan to answer the research questions. The
experiments  may mirror the research questions, or multiple research
questions (e.g., \rqref{rq:what} and \rqref{rq:three}) may be answered
by a single experiment. A simple approach for designing the evaluation
plan would be to design an experiment for testing each hypothesis, which
in turn will answer the research questions.

\subsection{Permissions Misuse Experiment}


    State some hypothesis for the experiment. Note which hypothesis and
    research question(s) it is designed to address.

    This experiment tests whether apps use all the permissions they request, whether dangerous permission combinations are used, and whether users are prompted and give consent to every permission used.

    \subsubsection{Experimental Setup}
    Describe {\em how} you are going to test the hypothesis. That is, what
    techniques/tools you are planning to use. Go into as much detail as
    possible. Be realistic in what you can achieve in the given time frame.

    We will scrape the MediaStore documentation for what permissions are associated with what constants (eg. CAMERA permission with ACTION_IMAGE_CAPTURE). After populating this dictionary, we will check to make sure that for every permission there is a corresponding constant found in the code, indicating that the permission has been used with MediaStore.

    We will use simple string searches in the Manifest to look for the permission combinations. 

    We will check for lines with .requestPermissions and then the name of each permission, to verify that requests are happening for each permission.


    \subsubsection{Expected Results}

    Describe the specific measurements and metrics you plan to use. Describe
    what constitutes success (i.e., what you expect to achieve).

    Criteria for success include:
    a MediaStore API call exists for all permissions requested, 
    no dangerous combinations are found, and
    a requestPermissions call exists for every permission.

    False positives would be encountered if 
    a dangerous combination is used, but there is no misuse of data, and both/all permissions in the combination are necessary to the app’s core functionality, or 
    non-MediaStore APIs are used with proper permissions.

\subsection{Name of Experiment 2}

    State some hypothesis for the experiment. Note which hypothesis and
    research question(s) it is designed to address.

    \subsubsection{Experimental Setup}
    Describe {\em how} you are going to test the hypothesis. That is, what
    techniques/tools you are planning to use. Go into as much detail as
    possible. Be realistic in what you can achieve in the given time frame.

    

    \subsubsection{Expected Results}

    Describe the specific measurements and metrics you plan to use. Describe
    what constitutes success (i.e., what you expect to achieve).

\subsection{Name of Experiment 3}

    State some hypothesis for the experiment. Note which hypothesis and
    research question(s) it is designed to address.

    \subsubsection{Experimental Setup}
    Describe {\em how} you are going to test the hypothesis. That is, what
    techniques/tools you are planning to use. Go into as much detail as
    possible. Be realistic in what you can achieve in the given time frame.

    \subsubsection{Expected Results}

    Describe the specific measurements and metrics you plan to use. Describe
    what constitutes success (i.e., what you expect to achieve).

\bibliographystyle{abbrv}
\bibliography{papers}

\end{document}


