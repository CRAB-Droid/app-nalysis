\documentclass[10pt,twocolumn,pdftex]{article}
\usepackage[margin=1in]{geometry}
\usepackage{comment}
\usepackage{graphicx}
\usepackage{url}
\usepackage[pdftex,colorlinks=true,citecolor=black,filecolor=black,%
            linkcolor=black,urlcolor=black]{hyperref}
\usepackage{times}
%\usepackage{listings}
%\usepackage{fancyvrb}
%\usepackage{amsmath}
%\usepackage{amsthm}
%\usepackage{amssymb}

%\lstset{ % for our code environment
%    language={},
%    basicstyle=\ttfamily}
%\let\code\lstinline

\title{CSCI 420-03: Analysis Plan}
\author{Student Name, Student Name \\
\url{{student.name, student.name}@email.wm.edu}}
\date{}

% Simple macro to help with research questions
\newcounter{rqcounter}
\setcounter{rqcounter}{0}
% Usage: \newrq{label}{question}
\newcommand{\newrq}[2]{\noindent\refstepcounter{rqcounter}\textbf{RQ\arabic{rqcounter}:} {\em #2}\label{#1}}
% Usage: \rqref{label}
\newcommand{\rqref}[1]{\textbf{RQ\ref{#1}}}


\begin{document}

\maketitle
\section{Research Questions}

A list of at least {\sf 9} (more desired) research questions (i.e., {\sf
3} per analysis goal).

\newrq{rq:what}{What are your research questions?}

\newrq{rq:three}{There should be at least nine, three per goal, but more
would be better. Do you have nine questions?}

\newrq{rq:specific}{Make sure the questions are specific, concrete, and
unambiguous. Are they?} 

Your analysis should answer all the research questions,
i.e.,~\rqref{rq:what}$\rightarrow$\rqref{rq:specific}.

\section{Hypotheses}

A high level description of how you plan to answer the research
questions, along with a list of hypotheses.

\section{Evaluation Plan}

A description of how you plan to answer the research questions. The
experiments  may mirror the research questions, or multiple research
questions (e.g., \rqref{rq:what} and \rqref{rq:three}) may be answered
by a single experiment. A simple approach for designing the evaluation
plan would be to design an experiment for testing each hypothesis, which
in turn will answer the research questions.

\subsection{Name of Experiment 1}

State some hypothesis for the experiment. Note which hypothesis and
research question(s) it is designed to address.

\subsubsection{Experimental Setup}
Describe {\em how} you are going to test the hypothesis. That is, what
techniques/tools you are planning to use. Go into as much detail as
possible. Be realistic in what you can achieve in the given time frame.

\subsubsection{Expected Results}

Describe the specific measurements and metrics you plan to use. Describe
what constitutes success (i.e., what you expect to achieve).

\subsection{Name of Experiment 2}

State some hypothesis for the experiment. Note which hypothesis and
research question(s) it is designed to address.

\subsubsection{Experimental Setup}
Describe {\em how} you are going to test the hypothesis. That is, what
techniques/tools you are planning to use. Go into as much detail as
possible. Be realistic in what you can achieve in the given time frame.

\subsubsection{Expected Results}

Describe the specific measurements and metrics you plan to use. Describe
what constitutes success (i.e., what you expect to achieve).

\subsection{Name of Experiment 3}

State some hypothesis for the experiment. Note which hypothesis and
research question(s) it is designed to address.

\subsubsection{Experimental Setup}
Describe {\em how} you are going to test the hypothesis. That is, what
techniques/tools you are planning to use. Go into as much detail as
possible. Be realistic in what you can achieve in the given time frame.

\subsubsection{Expected Results}

Describe the specific measurements and metrics you plan to use. Describe
what constitutes success (i.e., what you expect to achieve).

\bibliographystyle{abbrv}
\bibliography{papers}

\end{document}


